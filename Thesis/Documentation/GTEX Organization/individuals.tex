% Created 2014-09-27 Sat 14:09
\documentclass[11pt]{article}
\usepackage[utf8]{inputenc}
\usepackage[T1]{fontenc}
\usepackage{fixltx2e}
\usepackage{graphicx}
\usepackage{longtable}
\usepackage{float}
\usepackage{wrapfig}
\usepackage{soul}
\usepackage{textcomp}
\usepackage{marvosym}
\usepackage{wasysym}
\usepackage{latexsym}
\usepackage{amssymb}
\usepackage{hyperref}
\tolerance=1000
\usepackage{listings}
\providecommand{\alert}[1]{\textbf{#1}}

\title{This is a file that describes how to make the matrix that indicates individuals in rows and tissues in columns. It can then be used to generate the appropriate heatmap.}
\author{Sarah Urbut}
\date{\today}
\hypersetup{
  pdfkeywords={},
  pdfsubject={},
  pdfcreator={Emacs Org-mode version 7.9.3f}}

\begin{document}

\maketitle

\setcounter{tocdepth}{3}
\tableofcontents
\vspace*{1cm}


\lstset{breaklines,language=sh}
\begin{lstlisting}

cd individuals
for i in $( ls|grep "expr"); do
b=`basename $i ".expr.txt"`
echo $b
cat $i |head -1 > ./d/$b
done

for i in $(cat list.txt); do b=`basename $i ".expr.txt"`;echo $b; done
\end{lstlisting}


\begin{verbatim}
Adipose_Subcutaneous_Analysis
Adipose_Visceral_Omentum_Analysis
Adrenal_Gland_Analysis
Artery_Aorta_Analysis
Artery_Coronary_Analysis
Artery_Tibial_Analysis
Brain_Anterior_cingulate_cortex_BA24_Analysis
Brain_Caudate_basal_ganglia_Analysis
Brain_Cerebellar_Hemisphere_Analysis
Brain_Cerebellum_Analysis
Brain_Cortex_Analysis
Brain_Frontal_Cortex_BA9_Analysis
Brain_Hippocampus_Analysis
Brain_Hypothalamus_Analysis
Brain_Nucleus_accumbens_basal_ganglia_Analysis
Brain_Putamen_basal_ganglia_Analysis
Breast_Mammary_Tissue_Analysis
Cells_EBV-transformed_lymphocytes_Analysis
Cells_Transformed_fibroblasts_Analysis
Colon_Sigmoid_Analysis
Colon_Transverse_Analysis
\end{verbatim}




\lstset{breaklines,language=R}
\begin{lstlisting}
     setwd("./individuals/d")
      files=list.files()[-45]
      tissues=read.table("tissuenames.txt",row.names=NULL)
            names <- list()
            for(i in 1:length(files)){
                names[[i]]=data.frame(read.table(files[i],header=FALSE,sep="\t"))}

    ###To determine length:

       (max.inds <- max(unlist(lapply(names,function(x){ncol(x)}))))
    (tissue.max=tissues$V1[which.max(unlist(lapply(names,function(x){ncol(x)})))])
    namevector=t(data.frame(read.table(files[tissue.max],header=FALSE,sep="\t")))





    mat=matrix(NA,nrow=length(files),ncol=max.inds)

    for(i in 1:44){
        r=data.frame(names[[i]])
        l=ncol(r)

        if(l<=362)
            {
            nulls=362-l
            n=matrix(NA,ncol=nulls,nrow=1)
            r=as.matrix(cbind(r,n))
        }
            mat[i,]=r[1,]
    }


    rownames(mat)=tissues$V1

    ind.rows=t(mat)
    rownames(ind.rows)=namevector[,1]


    matched.mat=matrix(NA,ncol=44,nrow=362)
    for(i in 1:ncol(ind.rows)){
       matched.mat[,i]= ind.rows[match(namevector[,1],ind.rows[,i]),i]
    }

    rownames(matched.mat)=namevector[,1]
    colnames(matched.mat)=tissues$V1

  boolean.mat=matrix(NA,ncol=44,nrow=362)

    matched.mat[1:10,1:10]

  for(i in 1:nrow(matched.mat)){
      row=matched.mat[i,]
  t=sapply(row,function(x){
  if(is.na(x)){x=0}
    else if(!is.na(x)){x=1}
  })
    boolean.mat[i,]=t
  }

    rownames(boolean.mat)=namevector[,1]
    colnames(boolean.mat)=tissues$V1
 boolean.mat=data.frame(boolean.mat)
heatmap(boolean.mat)
\end{lstlisting}





\end{document}
